% !TEX program = xelatex
\documentclass[12pt,a4paper]{article}
\usepackage{fontspec}
\defaultfontfeatures{Mapping=tex-text}
\usepackage{xunicode}
\usepackage{xltxtra}
%\setmainfont{???}
\usepackage{amsmath}
\usepackage{blkarray}
\usepackage{amsfonts}
\usepackage{amssymb}
\usepackage{graphicx}
\usepackage[left=1.5cm,right=1.5cm,top=2.5cm,bottom=2.5cm]{geometry}

\usepackage{tikz}
\usepackage{tikz-qtree}


\usepackage[document]{ragged2e}


\usepackage{xcolor}
\definecolor{hlit}{rgb}{0.5,0,1}
\definecolor{diff}{rgb}{0.1,0.1,0.5}
\definecolor{rlit}{rgb}{1,0,0}

\title{NLP, Assignment 2}
\author{Zubair Abid, 20171076}
\date{}

\begin{document}
	\maketitle

	\section{Sentences}
	\begin{enumerate}
		\item The mother goes to school to drop the child.
		
		\item The mother goes to school for dropping the child.
		
		\item The mother goes to school having dropped the child.
		
	\end{enumerate}		
	
	\section{Phase Structure and Dependency Tree for each sentence}

	\begin{enumerate}
		\item The mother goes to school to drop the child.
		
		\Tree[.S	[.NP	
						[.det The ]
						[.Noun mother ] 
					]
					[.VP	
						[.VBZ goes ]
						[.PP
							[.P to ]
							[.NP [.Noun School ] ] ]
						[.S 
							[.VP
								[.TO to ]
								[.VP 
									[.Verb drop ]
									[.NP 
										[.det the ]
										[.Noun child ]
									]
								]
							]
						] 
					] 
			]		
		$ $\\
		$ $\\
		$ $\\
		\begin{tikzpicture}[level distance=1.5cm,sibling distance=3cm,]

		\Tree[.goes
						\edge node[auto=right] {$n-sub$}; 
						[.mother 
								\edge node[auto=left] {$det$};
								[.the ] 
						]
						\edge node[auto=right] {$x-comp$}; 
						[.drop	
								\edge node[auto=left] {$d-obj$};
								[.child 
									\edge node[auto=left] {$det$};
									[.the ]
								]
								\edge node[auto=left] {$aux$};
								[.to ] 
						]
						\edge node[auto=left] {$nmod$}; 
						[.school	
								\edge node[auto=left] {$case$};
								[.to ]
						]
		]		
		
		\end{tikzpicture}		
			
		$ $\\
		$ $\\
		$ $\\
				
		\item The mother goes to school for dropping the child.
		
		\Tree[.S	[.NP	
						[.det The ]
						[.Noun mother ] 
					]
					[.VP	
						[.VBZ goes ]
						[.PP
							[.P to ]
							[.NP [.Noun School ] ] ]
						[.S 
							[.VP
								[.IN for ]
								[.VP 
									[.Verb dropping ]
									[.NP 
										[.det the ]
										[.Noun child ]
									]
								]
							]
						] 
					] 
			]		
		$ $\\
		$ $\\
		$ $\\
		\begin{tikzpicture}[level distance=1.5cm,sibling distance=3cm,]

		\Tree[.goes
						\edge node[auto=right] {$n-sub$}; 
						[.mother 
								\edge node[auto=left] {$det$};
								[.the ] 
						]
						\edge node[auto=right] {$x-comp$}; 
						[.dropping	
								\edge node[auto=left] {$d-obj$};
								[.child 
									\edge node[auto=left] {$det$};
									[.the ]
								]
								\edge node[auto=left] {$aux$};
								[.for ] 
						]
						\edge node[auto=left] {$nmod$}; 
						[.school	
								\edge node[auto=left] {$case$};
								[.to ]
						]
		]		
		
		\end{tikzpicture}	
			
		$ $\\
		$ $\\
		$ $\\
		
		\item The mother goes to school having dropped the child.
				
		\Tree[.S	[.NP	
						[.det The ]
						[.Noun mother ] 
					]
					[.VP	
						[.VBZ goes ]
						[.PP
							[.P to ]
							[.NP [.Noun School ] ] ]
						[.S
							[.VP
								[.VBG having ]
								[.VP 
									[.Verb dropped ]
									[.NP 
										[.det the ]
										[.Noun child ]
									]
								]
							]
						] 
					] 
			]		
		$ $\\
		$ $\\
		$ $\\
		\begin{tikzpicture}[level distance=1.5cm,sibling distance=3cm,]

		\Tree[.goes
						\edge node[auto=right] {$n-sub$}; 
						[.mother 
								\edge node[auto=left] {$det$};
								[.the ] 
						]
						\edge node[auto=right] {$x-comp$}; 
						[.dropped	
								\edge node[auto=left] {$d-obj$};
								[.child 
									\edge node[auto=left] {$det$};
									[.the ]
								]
								\edge node[auto=left] {$aux$};
								[.having  ] 
						]
						\edge node[auto=left] {$nmod$}; 
						[.school	
								\edge node[auto=left] {$case$};
								[.to ]
						]
		]		
		
		\end{tikzpicture}	
			
		$ $\\
		$ $\\
		$ $\\
		
	\end{enumerate}	
	
	\section{Context Free Grammar rules to cover these}
    \begin{enumerate}

        \item S --> NP VP
        \item NP --> det Noun
        \item VP --> Verb NP
        \item VP --> TO VP
        \item VP --> VBZ PP S
        \item VP --> VBG VP
        \item VP --> IN VP
        \item PP --> P NP
        \item NP --> Noun
        \item S' --> VP

    \end{enumerate}

	
	\section{Feature Specifications covering different Tense-Aspects of the verb drop}
    \textbf{Drop} occurs in the different tense-aspects:\\
    \textit{drop}, \textit{dropped}, and \textit{dropping}

\[
\textsc{drop}=
\begin{bmatrix}
\textsc{root:} & \mspace{-12mu} \begin{matrix} \textbf{drop} \end{matrix} \\
\begin{matrix} \textsc{attr:} \end{matrix} &
\mspace{-12mu}
\begin{bmatrix}
  \textsc{tense:} & \textit{present}\\
  \textsc{case:} & \textit{to}\\
  \textsc{aspect:} & \textit{simple}\\
\end{bmatrix}
\end{bmatrix}
\]\\

\[
\textsc{dropped}=
\begin{bmatrix}
\textsc{root:} & \mspace{-12mu} \begin{matrix} \textbf{drop} \end{matrix} \\
\begin{matrix} \textsc{attr:} \end{matrix} &
\mspace{-12mu}
\begin{bmatrix}
  \textsc{tense:} & \textit{past}\\
  \textsc{case:} & \textit{to}\\
  \textsc{aspect:} & \textit{perfect}\\
\end{bmatrix}
\end{bmatrix}
\]\\
\[
\textsc{dropping}=
\begin{bmatrix}
\textsc{root:} & \mspace{-12mu} \begin{matrix} \textbf{drop} \end{matrix} \\
\begin{matrix} \textsc{attr:} \end{matrix} &
\mspace{-12mu}
\begin{bmatrix}
  \textsc{tense:} & \textit{continous}\\
  \textsc{case:} & \textit{for}\\
  \textsc{aspect:} & \textit{progressive}\\
\end{bmatrix}
\end{bmatrix}
\]
$ $\\
$ $\\
$ $\\
		This helps us deal with the number of rules we have, somewhat.\\
		((VP --> TO VP)), ((VP --> IN VP)), ((VP --> VBG VP)) get reduced.\\
		\[
\textsc{VP} ->
\begin{bmatrix} 
\textsc{root} \\
\textsc{case} \\
\textsc{aspect} \\
\end{bmatrix}
\]
		We unify the rules, as the general rule explains all the various ones and we can apply the correct one by looking up the feature specification table.\\

	
	\section{Handling agreement between mother and go using feature specifications}
	
	\[
\textsc{mother}=
\begin{bmatrix}
\textsc{root:} & \mspace{-12mu} \begin{matrix} \textbf{mother} \end{matrix} \\
\begin{matrix} \textsc{attr:} \end{matrix} &
\mspace{-12mu}
\begin{bmatrix}
  \textsc{gender:} & \textit{female}\\
  \textsc{number:} & \textit{singular}\\
\end{bmatrix}
\end{bmatrix}
\]\\
\[
\textsc{go}=
\begin{bmatrix}
\textsc{root:} & \mspace{-12mu} \begin{matrix} \textbf{go} \end{matrix} \\
\begin{matrix} \textsc{attr:} \end{matrix} &
\mspace{-12mu}
\begin{bmatrix}
  \textsc{tense:} & \textit{present}\\
  \textsc{aspect:} & \textit{simple}\\
\end{bmatrix}
\end{bmatrix}
\]
		Comparing against the forms used:\\
		
	\[
\textsc{the mother}=
\begin{bmatrix}
\textsc{specifier:} & \mspace{-12mu} \begin{matrix} \textbf{definite} \end{matrix} \\
\textsc{root:} & \mspace{-12mu} \begin{matrix} \textbf{mother} \end{matrix} \\
\begin{matrix} \textsc{attr:} \end{matrix} &
\mspace{-12mu}
\begin{bmatrix}
  \textsc{gender:} & \textit{female}\\
  \textsc{number:} & \textit{singular}\\
\end{bmatrix}
\end{bmatrix}
\]\\
\[
\textsc{goes}=
\begin{bmatrix}
\textsc{root:} & \mspace{-12mu} \begin{matrix} \textbf{go} \end{matrix} \\
\begin{matrix} \textsc{attr:} \end{matrix} &
\mspace{-12mu}
\begin{bmatrix}
  \textsc{tense:} & \textit{present}\\
  \textsc{aspect:} & \textit{simple}\\
\end{bmatrix}
\end{bmatrix}
\]
		As we see, the forms used ('the mother' and 'goes') inherit most of their features from the general FS for 'mother' and 'go'. They also agree in attributes.\\
		Observe the phrase "the mother goes"\\
			
			\[
\textsc{mother}=
\begin{bmatrix}
  \textsc{gender:} & \textit{female}\\
  \textsc{number:} & \textit{singular}\\
  \textsc{root:} & \textit{mother}\\
\end{bmatrix}
\]\\
\[
\textsc{the mother}=
\begin{bmatrix}
  \textsc{gender:} & \textit{female}\\
  \textsc{number:} & \textit{singular}\\
  \textsc{root:} & \textit{mother}\\
  \textsc{specifier:} & \textit{definite}\\
\end{bmatrix}
\]\\
\[
\textsc{the mother goes}=
\begin{bmatrix}
  \textsc{gender:} & \textit{female}\\
  \textsc{number:} & \textit{singular}\\
  \textsc{root:} & \textit{mother}\\
  \textsc{tense:} & \textit{present}\\
  \textsc{aspect:} & \textit{singular}\\
  \textsc{root:} & \textit{go}\\
\end{bmatrix}
\]
\textsl{•}
	
\end{document}
